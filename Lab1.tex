% Options for packages loaded elsewhere
\PassOptionsToPackage{unicode}{hyperref}
\PassOptionsToPackage{hyphens}{url}
%
\documentclass[
]{article}
\usepackage{lmodern}
\usepackage{amssymb,amsmath}
\usepackage{ifxetex,ifluatex}
\ifnum 0\ifxetex 1\fi\ifluatex 1\fi=0 % if pdftex
  \usepackage[T1]{fontenc}
  \usepackage[utf8]{inputenc}
  \usepackage{textcomp} % provide euro and other symbols
\else % if luatex or xetex
  \usepackage{unicode-math}
  \defaultfontfeatures{Scale=MatchLowercase}
  \defaultfontfeatures[\rmfamily]{Ligatures=TeX,Scale=1}
\fi
% Use upquote if available, for straight quotes in verbatim environments
\IfFileExists{upquote.sty}{\usepackage{upquote}}{}
\IfFileExists{microtype.sty}{% use microtype if available
  \usepackage[]{microtype}
  \UseMicrotypeSet[protrusion]{basicmath} % disable protrusion for tt fonts
}{}
\makeatletter
\@ifundefined{KOMAClassName}{% if non-KOMA class
  \IfFileExists{parskip.sty}{%
    \usepackage{parskip}
  }{% else
    \setlength{\parindent}{0pt}
    \setlength{\parskip}{6pt plus 2pt minus 1pt}}
}{% if KOMA class
  \KOMAoptions{parskip=half}}
\makeatother
\usepackage{xcolor}
\IfFileExists{xurl.sty}{\usepackage{xurl}}{} % add URL line breaks if available
\IfFileExists{bookmark.sty}{\usepackage{bookmark}}{\usepackage{hyperref}}
\hypersetup{
  pdftitle={Computer Lab 1},
  pdfauthor={Martynas Lukosevicius, Alejo Perez Gomez, Zahra Jalil Pour},
  hidelinks,
  pdfcreator={LaTeX via pandoc}}
\urlstyle{same} % disable monospaced font for URLs
\usepackage[margin=1in]{geometry}
\usepackage{color}
\usepackage{fancyvrb}
\newcommand{\VerbBar}{|}
\newcommand{\VERB}{\Verb[commandchars=\\\{\}]}
\DefineVerbatimEnvironment{Highlighting}{Verbatim}{commandchars=\\\{\}}
% Add ',fontsize=\small' for more characters per line
\usepackage{framed}
\definecolor{shadecolor}{RGB}{248,248,248}
\newenvironment{Shaded}{\begin{snugshade}}{\end{snugshade}}
\newcommand{\AlertTok}[1]{\textcolor[rgb]{0.94,0.16,0.16}{#1}}
\newcommand{\AnnotationTok}[1]{\textcolor[rgb]{0.56,0.35,0.01}{\textbf{\textit{#1}}}}
\newcommand{\AttributeTok}[1]{\textcolor[rgb]{0.77,0.63,0.00}{#1}}
\newcommand{\BaseNTok}[1]{\textcolor[rgb]{0.00,0.00,0.81}{#1}}
\newcommand{\BuiltInTok}[1]{#1}
\newcommand{\CharTok}[1]{\textcolor[rgb]{0.31,0.60,0.02}{#1}}
\newcommand{\CommentTok}[1]{\textcolor[rgb]{0.56,0.35,0.01}{\textit{#1}}}
\newcommand{\CommentVarTok}[1]{\textcolor[rgb]{0.56,0.35,0.01}{\textbf{\textit{#1}}}}
\newcommand{\ConstantTok}[1]{\textcolor[rgb]{0.00,0.00,0.00}{#1}}
\newcommand{\ControlFlowTok}[1]{\textcolor[rgb]{0.13,0.29,0.53}{\textbf{#1}}}
\newcommand{\DataTypeTok}[1]{\textcolor[rgb]{0.13,0.29,0.53}{#1}}
\newcommand{\DecValTok}[1]{\textcolor[rgb]{0.00,0.00,0.81}{#1}}
\newcommand{\DocumentationTok}[1]{\textcolor[rgb]{0.56,0.35,0.01}{\textbf{\textit{#1}}}}
\newcommand{\ErrorTok}[1]{\textcolor[rgb]{0.64,0.00,0.00}{\textbf{#1}}}
\newcommand{\ExtensionTok}[1]{#1}
\newcommand{\FloatTok}[1]{\textcolor[rgb]{0.00,0.00,0.81}{#1}}
\newcommand{\FunctionTok}[1]{\textcolor[rgb]{0.00,0.00,0.00}{#1}}
\newcommand{\ImportTok}[1]{#1}
\newcommand{\InformationTok}[1]{\textcolor[rgb]{0.56,0.35,0.01}{\textbf{\textit{#1}}}}
\newcommand{\KeywordTok}[1]{\textcolor[rgb]{0.13,0.29,0.53}{\textbf{#1}}}
\newcommand{\NormalTok}[1]{#1}
\newcommand{\OperatorTok}[1]{\textcolor[rgb]{0.81,0.36,0.00}{\textbf{#1}}}
\newcommand{\OtherTok}[1]{\textcolor[rgb]{0.56,0.35,0.01}{#1}}
\newcommand{\PreprocessorTok}[1]{\textcolor[rgb]{0.56,0.35,0.01}{\textit{#1}}}
\newcommand{\RegionMarkerTok}[1]{#1}
\newcommand{\SpecialCharTok}[1]{\textcolor[rgb]{0.00,0.00,0.00}{#1}}
\newcommand{\SpecialStringTok}[1]{\textcolor[rgb]{0.31,0.60,0.02}{#1}}
\newcommand{\StringTok}[1]{\textcolor[rgb]{0.31,0.60,0.02}{#1}}
\newcommand{\VariableTok}[1]{\textcolor[rgb]{0.00,0.00,0.00}{#1}}
\newcommand{\VerbatimStringTok}[1]{\textcolor[rgb]{0.31,0.60,0.02}{#1}}
\newcommand{\WarningTok}[1]{\textcolor[rgb]{0.56,0.35,0.01}{\textbf{\textit{#1}}}}
\usepackage{graphicx,grffile}
\makeatletter
\def\maxwidth{\ifdim\Gin@nat@width>\linewidth\linewidth\else\Gin@nat@width\fi}
\def\maxheight{\ifdim\Gin@nat@height>\textheight\textheight\else\Gin@nat@height\fi}
\makeatother
% Scale images if necessary, so that they will not overflow the page
% margins by default, and it is still possible to overwrite the defaults
% using explicit options in \includegraphics[width, height, ...]{}
\setkeys{Gin}{width=\maxwidth,height=\maxheight,keepaspectratio}
% Set default figure placement to htbp
\makeatletter
\def\fps@figure{htbp}
\makeatother
\setlength{\emergencystretch}{3em} % prevent overfull lines
\providecommand{\tightlist}{%
  \setlength{\itemsep}{0pt}\setlength{\parskip}{0pt}}
\setcounter{secnumdepth}{-\maxdimen} % remove section numbering

\title{Computer Lab 1}
\author{Martynas Lukosevicius, Alejo Perez Gomez, Zahra Jalil Pour}
\date{03/11/2020}

\begin{document}
\maketitle

\hypertarget{question-1-be-careful-when-comparing}{%
\subsection{Question 1 (Be Careful When
Comparing)}\label{question-1-be-careful-when-comparing}}

\hypertarget{section}{%
\subsubsection{1}\label{section}}

It is not possible to represent exact 1/3 and 1/12 in binary. As a
result it is rounded towards nearest computer float, which in R is equal
to 0.33333333333333331 and 0.083333333333333329 respectively. In the
first snippet code, the result is ``Subtraction is wrong'', but in the
second snippet code the result is ``Subtraction is correct''. A rational
number , of any size may or may not have an exact representation by a
floating point number. This is the familiar situation where fractions
such as 1/3 have no finite representation in base 10. The only numbers
that can be represented exactly in R are integers and powers of two.
Hence all other numbers are rounded to 53 binary digits accuracy.
Whenever floating point operations are done, we should assume that there
will be numeric error. 1/3 and 1/12 are repeating decimals that are
rounded in R.

\hypertarget{section-1}{%
\subsubsection{2}\label{section-1}}

Instead of writing \texttt{if(x1\ -\ x2\ ==\ 1/12)} it should be written
\texttt{if(isTRUE(all.equal(x1-x2,1/12)))}. In this case this equation
will return TRUE. We can use \texttt{all.equal} function , or we can use
\texttt{all.equal.numeric} function too.

\hypertarget{question-2-derivative}{%
\subsection{Question 2 (Derivative)}\label{question-2-derivative}}

\hypertarget{section-2}{%
\subsubsection{1}\label{section-2}}

Write your own R function to calculate the derivative of
\texttt{f(x)\ =\ x} in this way with \texttt{e\ =\ 10\^{}-15}.

\begin{Shaded}
\begin{Highlighting}[]
\NormalTok{  f <-}\StringTok{ }\ControlFlowTok{function}\NormalTok{(x)\{}
  \KeywordTok{return}\NormalTok{(x)}
\NormalTok{  \}}

\NormalTok{derivative <-}\StringTok{ }\ControlFlowTok{function}\NormalTok{(x,e)\{}
  \KeywordTok{return}\NormalTok{((}\KeywordTok{f}\NormalTok{(x}\OperatorTok{+}\NormalTok{e)}\OperatorTok{-}\KeywordTok{f}\NormalTok{(x))}\OperatorTok{/}\NormalTok{e)}
\NormalTok{\}}
\end{Highlighting}
\end{Shaded}

\hypertarget{section-3}{%
\subsubsection{2}\label{section-3}}

Evaluate your derivative function at \texttt{x\ =\ 1} and
\texttt{x\ =\ 100000}

\begin{Shaded}
\begin{Highlighting}[]
\NormalTok{e <-}\StringTok{ }\DecValTok{10}\OperatorTok{^}\NormalTok{(}\OperatorTok{-}\DecValTok{15}\NormalTok{)}
\NormalTok{x <-}\StringTok{ }\DecValTok{1}

\KeywordTok{derivative}\NormalTok{(x,e)}
\end{Highlighting}
\end{Shaded}

\begin{verbatim}
## [1] 1.110223
\end{verbatim}

\begin{Shaded}
\begin{Highlighting}[]
\NormalTok{e <-}\StringTok{ }\DecValTok{10}\OperatorTok{^}\NormalTok{(}\OperatorTok{-}\DecValTok{15}\NormalTok{)}
\NormalTok{x <-}\StringTok{ }\DecValTok{100000}

\KeywordTok{derivative}\NormalTok{(x,e)}
\end{Highlighting}
\end{Shaded}

\begin{verbatim}
## [1] 0
\end{verbatim}

\hypertarget{section-4}{%
\subsubsection{3}\label{section-4}}

When x = 1, derivative = 1.110223 and when x = 100000, derivative = 0

However, true values for both cases should be 1.

The smallest positive computer number is epsilon that here we considered
it to be\texttt{10\^{}(-15)}. When \texttt{x=100000} the derivative
function showed 0, in the equation\texttt{((x+e)-x)}, the difference
between large numbers dominates over epsilon, in other words the
smallest positive number is added to the large number. Hence the epsilon
would be ignored. However, when \texttt{x=1}, the effect of epsilon
cannot be ignored the result would be 1.110223.

\hypertarget{question-3-variance}{%
\subsection{Question 3 (Variance)}\label{question-3-variance}}

\hypertarget{section-5}{%
\subsubsection{1.}\label{section-5}}

\begin{Shaded}
\begin{Highlighting}[]
\NormalTok{myvar <-}\StringTok{ }\ControlFlowTok{function}\NormalTok{(x)\{}
\NormalTok{  n <-}\StringTok{ }\KeywordTok{length}\NormalTok{(x)}
\NormalTok{  xSq <-}\StringTok{ }\KeywordTok{sum}\NormalTok{(x}\OperatorTok{^}\DecValTok{2}\NormalTok{)}
\NormalTok{  sumXSq <-}\StringTok{ }\KeywordTok{sum}\NormalTok{(x)}\OperatorTok{^}\DecValTok{2}
\NormalTok{  part2 <-}\StringTok{ }\NormalTok{sumXSq}\OperatorTok{/}\NormalTok{n}
  \KeywordTok{return}\NormalTok{((xSq }\OperatorTok{-}\StringTok{ }\NormalTok{part2)}\OperatorTok{*}\StringTok{ }\NormalTok{(}\DecValTok{1}\OperatorTok{/}\NormalTok{(n}\DecValTok{-1}\NormalTok{)))}
\NormalTok{\}}
\end{Highlighting}
\end{Shaded}

\hypertarget{section-6}{%
\subsubsection{2}\label{section-6}}

\begin{Shaded}
\begin{Highlighting}[]
\NormalTok{x <-}\StringTok{ }\KeywordTok{rnorm}\NormalTok{(}\DecValTok{10000}\NormalTok{, }\DecValTok{10}\OperatorTok{^}\DecValTok{8}\NormalTok{, }\DecValTok{1}\NormalTok{)}
\end{Highlighting}
\end{Shaded}

\hypertarget{section-7}{%
\subsubsection{3}\label{section-7}}

\begin{Shaded}
\begin{Highlighting}[]
\NormalTok{result <-}\StringTok{ }\KeywordTok{list}\NormalTok{()}
\KeywordTok{options}\NormalTok{(}\DataTypeTok{digits =} \DecValTok{22}\NormalTok{ )}
\ControlFlowTok{for}\NormalTok{ (i }\ControlFlowTok{in} \DecValTok{1}\OperatorTok{:}\KeywordTok{length}\NormalTok{(x)) \{}
\NormalTok{  temp <-}\StringTok{ }\NormalTok{x[}\DecValTok{1}\OperatorTok{:}\NormalTok{i]}
\NormalTok{  y <-}\StringTok{ }\KeywordTok{myvar}\NormalTok{(temp) }\OperatorTok{-}\StringTok{ }\KeywordTok{var}\NormalTok{(temp)}
\NormalTok{  result <-}\StringTok{ }\KeywordTok{append}\NormalTok{(result, y)}
\NormalTok{\}}

\KeywordTok{plot}\NormalTok{(}\KeywordTok{c}\NormalTok{(}\DecValTok{1}\OperatorTok{:}\KeywordTok{length}\NormalTok{(x)), result, }\DataTypeTok{main =} \StringTok{"myvar() - var()"}\NormalTok{,}
     \DataTypeTok{xlab =} \StringTok{"i"}\NormalTok{,}\DataTypeTok{type =} \StringTok{"p"}\NormalTok{,  }\DataTypeTok{pch =} \DecValTok{20}\NormalTok{ , }\DataTypeTok{ylab =} \StringTok{"difference"}\NormalTok{)}
\end{Highlighting}
\end{Shaded}

\includegraphics{Lab1_files/figure-latex/unnamed-chunk-7-1.pdf}

The function does not work properly. It oscillates primarily when a
smaller set of values is involved in the variances calculation. It can
also be noted that the oscillation pattern in the produced response
(myVar(Xi) - var(Xi)) decreases as the value of terms involved
increases. Squaring big numbers result in overflow, since the computer
is not able to handle such a large number correctly. That is because the
numbers are so big that the computer cannot destine the right amount of
bytes for them. Moreover, first squaring and summing might be less than
first summing and later squaring. This is why our function will not
produce correct answers.

\hypertarget{section-8}{%
\subsubsection{4}\label{section-8}}

\begin{Shaded}
\begin{Highlighting}[]
\NormalTok{myvar2 <-}\StringTok{ }\ControlFlowTok{function}\NormalTok{(x)\{}
\NormalTok{  n <-}\StringTok{ }\KeywordTok{length}\NormalTok{(x)}
  \KeywordTok{return}\NormalTok{((}\KeywordTok{sum}\NormalTok{((x }\OperatorTok{-}\StringTok{ }\KeywordTok{mean}\NormalTok{(x))}\OperatorTok{^}\DecValTok{2}\NormalTok{))}\OperatorTok{/}\NormalTok{(n}\DecValTok{-1}\NormalTok{))}
\NormalTok{\}}

\NormalTok{result <-}\StringTok{ }\KeywordTok{list}\NormalTok{()}
\KeywordTok{options}\NormalTok{(}\DataTypeTok{digits =} \DecValTok{22}\NormalTok{ )}
\ControlFlowTok{for}\NormalTok{ (i }\ControlFlowTok{in} \DecValTok{1}\OperatorTok{:}\KeywordTok{length}\NormalTok{(x)) \{}
\NormalTok{  temp <-}\StringTok{ }\NormalTok{x[}\DecValTok{1}\OperatorTok{:}\NormalTok{i]}
\NormalTok{  y <-}\StringTok{ }\KeywordTok{myvar2}\NormalTok{(temp) }\OperatorTok{-}\StringTok{ }\KeywordTok{var}\NormalTok{(temp)}
\NormalTok{  result <-}\StringTok{ }\KeywordTok{append}\NormalTok{(result, y)}
\NormalTok{\}}

\KeywordTok{plot}\NormalTok{(}\KeywordTok{c}\NormalTok{(}\DecValTok{1}\OperatorTok{:}\KeywordTok{length}\NormalTok{(x)), result, }\DataTypeTok{main =} \StringTok{"myvar2() - var()"}\NormalTok{, }\DataTypeTok{xlab =} \StringTok{"i"}\NormalTok{,}\DataTypeTok{type =} \StringTok{"p"}\NormalTok{,}
     \DataTypeTok{pch =} \DecValTok{20}\NormalTok{ , }\DataTypeTok{ylab =} \StringTok{"difference"}\NormalTok{, }\DataTypeTok{ylim =} \KeywordTok{c}\NormalTok{(}\OperatorTok{-}\DecValTok{5}\NormalTok{,}\DecValTok{5}\NormalTok{))}
\end{Highlighting}
\end{Shaded}

\includegraphics{Lab1_files/figure-latex/unnamed-chunk-8-1.pdf}

\hypertarget{question-4-binomial-coeficient}{%
\subsection{Question 4 (Binomial
coeficient)}\label{question-4-binomial-coeficient}}

\hypertarget{section-9}{%
\subsubsection{1}\label{section-9}}

A: n , k and n - k cant be zero

B: n and n - k cant be zero

C: same as B

because prod(0) = 0 and 0 / 0 will be NaN

\hypertarget{section-10}{%
\subsubsection{2}\label{section-10}}

\includegraphics{Lab1_files/figure-latex/unnamed-chunk-10-1.pdf}

\includegraphics{Lab1_files/figure-latex/unnamed-chunk-11-1.pdf}

\hypertarget{section-11}{%
\subsubsection{3}\label{section-11}}

The expressions A and B, because with large numbers method prod() will
overflow.

In expression A we calculate the product of a vector from 1 to n and
later divide it by other products with smaller vectors. However, in this
case, the first operation (prod(1:n)) will overflow ( = Inf) and other
operations won't matter as the result will be Inf or Nan (if denominator
will be also Inf).

In expression B overflow will depend on k, if k is close to n it won't
overflow.

In expression C, as first vectors are divided, the final vector for
product will have smaller values and that is why prod() method won't
overflow.

\newpage

\hypertarget{appendix}{%
\section{Appendix}\label{appendix}}

\begin{Shaded}
\begin{Highlighting}[]
\CommentTok{#1}
\CommentTok{# all.equal.numeric() and isTRUE() function}
\NormalTok{x1 <-}\StringTok{ }\DecValTok{1}\OperatorTok{/}\DecValTok{3}
\NormalTok{x2 <-}\StringTok{ }\DecValTok{1}\OperatorTok{/}\DecValTok{4}
\ControlFlowTok{if}\NormalTok{ (}\KeywordTok{isTRUE}\NormalTok{(}\KeywordTok{all.equal.numeric}\NormalTok{(x1}\OperatorTok{-}\NormalTok{x2, }\DecValTok{1}\OperatorTok{/}\DecValTok{12}\NormalTok{))) \{}
  \KeywordTok{print}\NormalTok{ (}\StringTok{"Subtraction is correct"}\NormalTok{ )}
\NormalTok{\} }\ControlFlowTok{else}\NormalTok{ \{}
  \KeywordTok{print}\NormalTok{ (}\StringTok{"Subtraction is wrong"}\NormalTok{)}
\NormalTok{\}}

\CommentTok{#2}

\NormalTok{f <-}\StringTok{ }\ControlFlowTok{function}\NormalTok{(x)\{}
  \KeywordTok{return}\NormalTok{(x)}
\NormalTok{\}}

\NormalTok{derivative <-}\StringTok{ }\ControlFlowTok{function}\NormalTok{(x,e)\{}
  \KeywordTok{return}\NormalTok{((}\KeywordTok{f}\NormalTok{(x}\OperatorTok{+}\NormalTok{e)}\OperatorTok{-}\KeywordTok{f}\NormalTok{(x))}\OperatorTok{/}\NormalTok{e)}
\NormalTok{\}}

\NormalTok{e <-}\StringTok{ }\DecValTok{10}\OperatorTok{^}\NormalTok{(}\OperatorTok{-}\DecValTok{15}\NormalTok{)}
\NormalTok{x <-}\StringTok{ }\DecValTok{1}

\KeywordTok{derivative}\NormalTok{(x,e)}

\NormalTok{e <-}\StringTok{ }\DecValTok{10}\OperatorTok{^}\NormalTok{(}\OperatorTok{-}\DecValTok{15}\NormalTok{)}
\NormalTok{x <-}\StringTok{ }\DecValTok{100000}

\KeywordTok{derivative}\NormalTok{(x,e)}

\CommentTok{#3}
\NormalTok{myvar <-}\StringTok{ }\ControlFlowTok{function}\NormalTok{(x)\{}
\NormalTok{  n <-}\StringTok{ }\KeywordTok{length}\NormalTok{(x)}
\NormalTok{  xSq <-}\StringTok{ }\KeywordTok{sum}\NormalTok{(x}\OperatorTok{^}\DecValTok{2}\NormalTok{)}
\NormalTok{  sumXSq <-}\StringTok{ }\KeywordTok{sum}\NormalTok{(x)}\OperatorTok{^}\DecValTok{2}
\NormalTok{  part2 <-}\StringTok{ }\NormalTok{sumXSq}\OperatorTok{/}\NormalTok{n}
  \KeywordTok{return}\NormalTok{((xSq }\OperatorTok{-}\StringTok{ }\NormalTok{part2)}\OperatorTok{*}\StringTok{ }\NormalTok{(}\DecValTok{1}\OperatorTok{/}\NormalTok{(n}\DecValTok{-1}\NormalTok{)))}
\NormalTok{\}}

\NormalTok{x <-}\StringTok{ }\KeywordTok{rnorm}\NormalTok{(}\DecValTok{10000}\NormalTok{, }\DecValTok{10}\OperatorTok{^}\DecValTok{8}\NormalTok{, }\DecValTok{1}\NormalTok{)}

\NormalTok{result <-}\StringTok{ }\KeywordTok{list}\NormalTok{()}
\KeywordTok{options}\NormalTok{(}\DataTypeTok{digits =} \DecValTok{22}\NormalTok{ )}
\ControlFlowTok{for}\NormalTok{ (i }\ControlFlowTok{in} \DecValTok{1}\OperatorTok{:}\KeywordTok{length}\NormalTok{(x)) \{}
\NormalTok{  temp <-}\StringTok{ }\NormalTok{x[}\DecValTok{1}\OperatorTok{:}\NormalTok{i]}
\NormalTok{  y <-}\StringTok{ }\KeywordTok{myvar}\NormalTok{(temp) }\OperatorTok{-}\StringTok{ }\KeywordTok{var}\NormalTok{(temp)}
\NormalTok{  result <-}\StringTok{ }\KeywordTok{append}\NormalTok{(result, y)}
\NormalTok{\}}

\KeywordTok{plot}\NormalTok{(}\KeywordTok{c}\NormalTok{(}\DecValTok{1}\OperatorTok{:}\KeywordTok{length}\NormalTok{(x)), result, }\DataTypeTok{main =} \StringTok{"myvar() - var()"}\NormalTok{,}
     \DataTypeTok{xlab =} \StringTok{"i"}\NormalTok{,}\DataTypeTok{type =} \StringTok{"p"}\NormalTok{,  }\DataTypeTok{pch =} \DecValTok{20}\NormalTok{ , }\DataTypeTok{ylab =} \StringTok{"difference"}\NormalTok{)}

\NormalTok{myvar2 <-}\StringTok{ }\ControlFlowTok{function}\NormalTok{(x)\{}
\NormalTok{  n <-}\StringTok{ }\KeywordTok{length}\NormalTok{(x)}
  \KeywordTok{return}\NormalTok{((}\KeywordTok{sum}\NormalTok{((x }\OperatorTok{-}\StringTok{ }\KeywordTok{mean}\NormalTok{(x))}\OperatorTok{^}\DecValTok{2}\NormalTok{))}\OperatorTok{/}\NormalTok{(n}\DecValTok{-1}\NormalTok{))}
\NormalTok{\}}

\NormalTok{result <-}\StringTok{ }\KeywordTok{list}\NormalTok{()}
\KeywordTok{options}\NormalTok{(}\DataTypeTok{digits =} \DecValTok{22}\NormalTok{ )}
\ControlFlowTok{for}\NormalTok{ (i }\ControlFlowTok{in} \DecValTok{1}\OperatorTok{:}\KeywordTok{length}\NormalTok{(x)) \{}
\NormalTok{  temp <-}\StringTok{ }\NormalTok{x[}\DecValTok{1}\OperatorTok{:}\NormalTok{i]}
\NormalTok{  y <-}\StringTok{ }\KeywordTok{myvar2}\NormalTok{(temp) }\OperatorTok{-}\StringTok{ }\KeywordTok{var}\NormalTok{(temp)}
\NormalTok{  result <-}\StringTok{ }\KeywordTok{append}\NormalTok{(result, y)}
\NormalTok{\}}

\KeywordTok{plot}\NormalTok{(}\KeywordTok{c}\NormalTok{(}\DecValTok{1}\OperatorTok{:}\KeywordTok{length}\NormalTok{(x)), result, }\DataTypeTok{main =} \StringTok{"myvar2() - var()"}\NormalTok{, }\DataTypeTok{xlab =} \StringTok{"i"}\NormalTok{,}\DataTypeTok{type =} \StringTok{"p"}\NormalTok{,}
     \DataTypeTok{pch =} \DecValTok{20}\NormalTok{ , }\DataTypeTok{ylab =} \StringTok{"difference"}\NormalTok{, }\DataTypeTok{ylim =} \KeywordTok{c}\NormalTok{(}\OperatorTok{-}\DecValTok{5}\NormalTok{,}\DecValTok{5}\NormalTok{))}

\CommentTok{#4}

\NormalTok{n <-}\StringTok{ }\DecValTok{0}
\NormalTok{k <-}\StringTok{ }\DecValTok{0}
\KeywordTok{prod}\NormalTok{(}\DecValTok{1}\OperatorTok{:}\NormalTok{n) }\OperatorTok{/}\StringTok{ }\NormalTok{(}\KeywordTok{prod}\NormalTok{(}\DecValTok{1}\OperatorTok{:}\NormalTok{k) }\OperatorTok{*}\StringTok{ }\KeywordTok{prod}\NormalTok{(}\DecValTok{1}\OperatorTok{:}\NormalTok{(n}\OperatorTok{-}\NormalTok{k)))}
\KeywordTok{prod}\NormalTok{((k}\OperatorTok{+}\DecValTok{1}\NormalTok{)}\OperatorTok{:}\NormalTok{n) }\OperatorTok{/}\StringTok{ }\KeywordTok{prod}\NormalTok{(}\DecValTok{1}\OperatorTok{:}\NormalTok{(n}\OperatorTok{-}\NormalTok{k))}
\KeywordTok{prod}\NormalTok{(((k}\OperatorTok{+}\DecValTok{1}\NormalTok{)}\OperatorTok{:}\NormalTok{n) }\OperatorTok{/}\StringTok{ }\NormalTok{(}\DecValTok{1}\OperatorTok{:}\NormalTok{(n}\OperatorTok{-}\NormalTok{k)))}


\CommentTok{#n <- c(20:100)}

\NormalTok{calc1 <-}\StringTok{ }\ControlFlowTok{function}\NormalTok{(n,k)\{}
\NormalTok{k <-}\StringTok{ }\NormalTok{n }\OperatorTok{-}\StringTok{ }\NormalTok{k}
 \KeywordTok{return}\NormalTok{(}\KeywordTok{prod}\NormalTok{(}\DecValTok{1}\OperatorTok{:}\NormalTok{n) }\OperatorTok{/}\StringTok{ }\NormalTok{(}\KeywordTok{prod}\NormalTok{(}\DecValTok{1}\OperatorTok{:}\NormalTok{k) }\OperatorTok{*}\StringTok{ }\KeywordTok{prod}\NormalTok{(}\DecValTok{1}\OperatorTok{:}\NormalTok{(n}\OperatorTok{-}\NormalTok{k)))) }
\NormalTok{\}}
\NormalTok{calc2 <-}\StringTok{ }\ControlFlowTok{function}\NormalTok{(n,k)\{}
\NormalTok{  k <-}\StringTok{ }\NormalTok{n }\OperatorTok{-}\StringTok{ }\NormalTok{k}
 \KeywordTok{return}\NormalTok{(}\KeywordTok{prod}\NormalTok{((k}\OperatorTok{+}\DecValTok{1}\NormalTok{)}\OperatorTok{:}\NormalTok{n) }\OperatorTok{/}\StringTok{ }\KeywordTok{prod}\NormalTok{(}\DecValTok{1}\OperatorTok{:}\NormalTok{(n}\OperatorTok{-}\NormalTok{k))) }
\NormalTok{\}}
\NormalTok{calc3 <-}\StringTok{ }\ControlFlowTok{function}\NormalTok{(n,k)\{}
\NormalTok{  k <-}\StringTok{ }\NormalTok{n }\OperatorTok{-}\StringTok{ }\NormalTok{k}
 \KeywordTok{return}\NormalTok{(}\KeywordTok{prod}\NormalTok{(((k}\OperatorTok{+}\DecValTok{1}\NormalTok{)}\OperatorTok{:}\NormalTok{n) }\OperatorTok{/}\StringTok{ }\NormalTok{(}\DecValTok{1}\OperatorTok{:}\NormalTok{(n}\OperatorTok{-}\NormalTok{k)))) }
\NormalTok{\}}

\NormalTok{n <-}\StringTok{ }\KeywordTok{c}\NormalTok{(}\DecValTok{20}\OperatorTok{:}\DecValTok{400}\NormalTok{)}
\NormalTok{k <-}\StringTok{ }\DecValTok{20}

\NormalTok{y <-}\StringTok{  }\KeywordTok{sapply}\NormalTok{(n, calc1, }\DataTypeTok{k =}\NormalTok{ k)}
\NormalTok{y2 <-}\StringTok{  }\KeywordTok{sapply}\NormalTok{(n, calc2, }\DataTypeTok{k =}\NormalTok{ k)}
\NormalTok{y3 <-}\StringTok{  }\KeywordTok{sapply}\NormalTok{(n, calc3, }\DataTypeTok{k =}\NormalTok{ k)}
\KeywordTok{par}\NormalTok{(}\DataTypeTok{mfrow=}\KeywordTok{c}\NormalTok{(}\DecValTok{3}\NormalTok{,}\DecValTok{1}\NormalTok{))}
\KeywordTok{plot}\NormalTok{(n,y, }\DataTypeTok{type =} \StringTok{"l"}\NormalTok{, }\DataTypeTok{main =} \StringTok{"k = n - 20"}\NormalTok{, }\DataTypeTok{sub =}  \StringTok{"A"}\NormalTok{  , }\DataTypeTok{ylab =} \StringTok{"y"}\NormalTok{)}
\KeywordTok{plot}\NormalTok{(n,y2, }\DataTypeTok{type =} \StringTok{"l"}\NormalTok{, }\DataTypeTok{sub =}  \StringTok{"B"}\NormalTok{ , }\DataTypeTok{ylab =} \StringTok{"y"}\NormalTok{)}
\KeywordTok{plot}\NormalTok{(n, y3, }\DataTypeTok{type =} \StringTok{"l"}\NormalTok{, }\DataTypeTok{sub =}  \StringTok{"C"}\NormalTok{ , }\DataTypeTok{ylab =} \StringTok{"y"}\NormalTok{)}

\NormalTok{calc1 <-}\StringTok{ }\ControlFlowTok{function}\NormalTok{(n,k)\{}
 \KeywordTok{return}\NormalTok{(}\KeywordTok{prod}\NormalTok{(}\DecValTok{1}\OperatorTok{:}\NormalTok{n) }\OperatorTok{/}\StringTok{ }\NormalTok{(}\KeywordTok{prod}\NormalTok{(}\DecValTok{1}\OperatorTok{:}\NormalTok{k) }\OperatorTok{*}\StringTok{ }\KeywordTok{prod}\NormalTok{(}\DecValTok{1}\OperatorTok{:}\NormalTok{(n}\OperatorTok{-}\NormalTok{k)))) }
\NormalTok{\}}
\NormalTok{calc2 <-}\StringTok{ }\ControlFlowTok{function}\NormalTok{(n,k)\{}
 \KeywordTok{return}\NormalTok{(}\KeywordTok{prod}\NormalTok{((k}\OperatorTok{+}\DecValTok{1}\NormalTok{)}\OperatorTok{:}\NormalTok{n) }\OperatorTok{/}\StringTok{ }\KeywordTok{prod}\NormalTok{(}\DecValTok{1}\OperatorTok{:}\NormalTok{(n}\OperatorTok{-}\NormalTok{k))) }
\NormalTok{\}}
\NormalTok{calc3 <-}\StringTok{ }\ControlFlowTok{function}\NormalTok{(n,k)\{}
 \KeywordTok{return}\NormalTok{(}\KeywordTok{prod}\NormalTok{(((k}\OperatorTok{+}\DecValTok{1}\NormalTok{)}\OperatorTok{:}\NormalTok{n) }\OperatorTok{/}\StringTok{ }\NormalTok{(}\DecValTok{1}\OperatorTok{:}\NormalTok{(n}\OperatorTok{-}\NormalTok{k)))) }
\NormalTok{\}}

\NormalTok{n <-}\StringTok{ }\KeywordTok{c}\NormalTok{(}\DecValTok{20}\OperatorTok{:}\DecValTok{400}\NormalTok{)}
\NormalTok{k <-}\StringTok{ }\DecValTok{20}

\NormalTok{y <-}\StringTok{  }\KeywordTok{sapply}\NormalTok{(n, calc1, }\DataTypeTok{k =}\NormalTok{ k)}
\NormalTok{y2 <-}\StringTok{  }\KeywordTok{sapply}\NormalTok{(n, calc2, }\DataTypeTok{k =}\NormalTok{ k)}
\NormalTok{y3 <-}\StringTok{  }\KeywordTok{sapply}\NormalTok{(n, calc3, }\DataTypeTok{k =}\NormalTok{ k)}
\KeywordTok{par}\NormalTok{(}\DataTypeTok{mfrow=}\KeywordTok{c}\NormalTok{(}\DecValTok{3}\NormalTok{,}\DecValTok{1}\NormalTok{))}

\KeywordTok{plot}\NormalTok{(n,y, }\DataTypeTok{type =} \StringTok{"l"}\NormalTok{, }\DataTypeTok{main =} \StringTok{"k = 20"}\NormalTok{, }\DataTypeTok{sub =}  \StringTok{"A  "}\NormalTok{ , }\DataTypeTok{ylab =} \StringTok{"y"}\NormalTok{)}
\KeywordTok{plot}\NormalTok{(n,y2, }\DataTypeTok{type =} \StringTok{"l"}\NormalTok{, }\DataTypeTok{sub =} \StringTok{"B"}\NormalTok{, }\DataTypeTok{ylab =} \StringTok{"y"}\NormalTok{)}
\KeywordTok{plot}\NormalTok{(n, y3, }\DataTypeTok{type =} \StringTok{"l"}\NormalTok{, }\DataTypeTok{sub =} \StringTok{"C"}\NormalTok{, }\DataTypeTok{ylab =} \StringTok{"y"}\NormalTok{)}
\end{Highlighting}
\end{Shaded}

\end{document}
